
\section{Introduction}

More computing is being performed in web browsers. Games, multimedia, 
image and video processing, and other compute-intensive applications are 
being ported to JavaScript to be made available on the Web. An increasingly 
popular approach for porting existing native applications to the web is to use
Mozilla's Emscripten LLVM-to-JavaScript compiler. Emscripten compiles
C/C++ code into highly optimizable JavaScript that runs at near-native speeds
without the use of any plugins.

While these applications can exploit the use of vector code when written in C/C++, 
JavaScript applications were not able to make full use of the high performant and 
energy efficient SIMD instructions which are now standard on modern x86 and 
ARM hardware from mobile to servers. With the recent introduction of the portable 
language extension SIMD.JS, JavaScript applications are now able to use the 
provided SIMD primitives and operations to improve its performance by making full 
use of the SIMD hardware available.

This paper explores the implementation and evaluation of Emscripten's ability to generate 
SIMD.JS code from its native C/C++ intrinsics counterpart. We present performance 
data of the SIMD.JS benchmarks in JavaScript that have been automatically generated
by Emscripten and compare it with its hand-written version. We use both of the 
available prototypes of the SIMD.JS language extension: Google's V8 and Mozilla's 
SpiderMonkey and show speedup in both.

